\lesson{5}{05.10.2023}{Продолжение}

\begin{proof} (Продолжение доказательства)
    


    \[x_{n+1} = 2 + \sum_{k = 2}^{n} \frac{1}{k!} (1 - \frac{k - 1}{n + 1}) \cdot \ldots \cdot (1 - \frac{1}{n + 1}) + \frac{1}{(n+1)!} (1 - \frac{n}{n+1})\cdot \ldots \cdot (1 - \frac{1}{n+ 1}) \eqno(2) \]


    \[ \forall r > 0: 1 - \frac{r}{n+1} > 1 - \frac{r}{n} \implies (1 - \frac{k - 1}{n + 1}) \cdot \ldots \cdot (1 - \frac{1}{n + 1}) > (1 - \frac{k - 1}{n}) \cdot \ldots \cdot (1 - \frac{1}{n}) \eqno(1)\]  


    $(1), (2) \implies x_{n+1} > x_{n}$  

    Примем во внимание неравенства для $y_n$ и неравенства для $x_n$. Тогда мы будем иметь следующее неравенство:

    \[\Rightarrow x_1 < x_2 < \dots < x_n < y_n < y_{n - 1} < \dots < y_1 \eqno(4)\]

    \[(4) \Rightarrow x_n < y_1, y_n > x, \forall n \eqno(5)\]

    Последовательность $x_n$ строго возрастает и ограниченна сверху. Мы можем применить критерий существования конечного предела у строго монотонной возрастающей последовательности.

    \[\exists \lim_{n \to \infty} x_n = a \]

    Если мы посмотрим на последовательность $y_n$, она ограничена снизу в отношении пять и мы знаем что она строго монотонно убвает. По теореме о предельной последовательности получаем, что:


    \[\exists \lim_{n \ to \infty} y_n = b \]

    Теперь,

    \[ b = \lim_{n \to \infty} y_n = \lim_{n \to \infty} (1 + \frac{1}{n})^{n + 1} = \]



    \[ \lim_{n \to \infty} (1 + \frac{1}{n}) \cdot \lim_{n \to \infty} (1 + \frac{1}{n})^n = 1 + \cdot \lim_{n \to \infty} x_n = a\]

    Таким образом,
    \[a = b = e \eqno(6)\].


    \[ (6) \Rightarrow x_n < e < y_n \forall n \eqno(7)\]

    \[ (7) \Rightarrow e > x_1 = 2, e < y_5 < 3 \]

    \[ y_5 = (\frac{6}{5})^6 \]

    \[ e = 2.718... \]
\end{proof}



\begin{remark}
    Число e --- одно из фундаментальных констант на которой держится вся математика.

    Первые две - это 0 и 1. А третья --- это $\pi$


\end{remark}

\section{Критерий Коши, существование конечного предела последовательности}

\begin{theorem}

    Пусть имеется некоторая последовательность $\{x_n\}_{n = 1}^\infty$.
    Для того чтобы $\exists \lim_{n \to \infty} x_n \in \R$ необходимо и достаточно, чтобы $\forall \epsilon > 0, \exists N: \forall m, \forall n > N:$
    \[|x_m - x_m| < \epsilon \eqno(8)\]
\end{theorem}

\begin{remark}
    В формулировке не сказано чему будет равен этот предел. Какой именно он будет - неизвесто. Известно только то что он существует. 
    Это так называемая теорема существования.
\end{remark}

\begin{note}
    Необходимость означает что предел существует.
\end{note}
\begin{proof}

    
    Докажем необходимость. Предположим, что


    \[ \lim_{k \to \infty} x_k = a \in \R \]

    Тогда, по определению предела для любого $\epsilon > 0 \exists N$ такой, что $\forall n > N$ выполнено
    \[|x_n - a| < \frac{\epsilon}{2} \eqno(9)\]

    Тогда,

    \[ (9) \Rightarrow \text{при } n > N, m > N \]
    \[ |x_m - x_n| = |(x_m - a) - (x_n - a)| \leq |x_m - a| + |x_n - a| < \frac{\epsilon}{2} + \frac{\epsilon}{2} = \epsilon \Rightarrow (8)\]

    То-есть, необходимость доказана. Если конечный предел существует, то соотношение 8 выполнено.

    $\newline$    
    
    Теперь докажем достаточность.

    Когда мы будем доказывать достаточность, то мы не знаем, существует предел или нет.

    \begin{remark}
        Не каждая последователность имеет предел (например, $x_n = -1^n$).
    \end{remark}

    Для доказательства мы будем использовать теорему Дедекинда.

    Определим сечение множества вещественных чисел.

    Нижний класс A --- это

    \[ A = \{\alpha \in \R: \exists N: \forall n > N: x_n > \alpha\} \eqno(10) \]


    Вернхний класс A`' --- это

    \[ A' = \R \setminus A \eqno(10') \]

    Множества, получившиеся в (10) и (10') - это сечения, и это нужно проверить.

    \begin{itemize}
        \item Возьмём $\epsilon = 1$, тогда: $\exists N_0: \forall m, n > N_0: |x_m - x_n| < 1$

        В частности, при $m = N + 1$ и при $n > N + 1$ имеем
        \[|x_n - x_{N + 1}| < 1 \Leftrightarrow x_{N + 1} - 1 < x_n < x_{N + 1} + 1 \eqno(11) \]

        \[ (11) \implies x_{N+1} - 1 \in A \eqno(12) \]

        С другой стороны,
        \[ (11) \Rightarrow x_{N + 1} + 1 \notin A, \text{ то-есть, } x_{N + 1} + 1 \in A' \eqno(13) \]

        \[ (12), (13) \Rightarrow A \neq \varnothing, A' \neq \varnothing \]

        \item Никакое из них не может быть множеством вещественных чисел.



        Давайте возьмём $\forall \alpha \in A, \forall \beta \in A'$. Нужно доказать, что $\alpha$ всегда меньше $\beta$. В этом состоит условие определения сечения.

        \[ \alpha \in A =(10)> \exists N: \forall n > N x_n > \alpha \eqno(14) \]

        Если бы для любого $\forall n > N$ выполнялось $x_n > \beta$, то $\beta \in A$. Однако, это не так, т.к. $\beta \in A'$.

        То-есть,
        \[\exists n_0 > N: x_{n_0} \leq \beta \eqno(15)\]

        \begin{note}
            Если бы всё время неравенство было в другую сторону ($x_n > \beta$), тогда бы по определению (10), мы бы получили, что $\beta \in A$, но мы взяли $\beta \in A'$, то есть $\beta \notin A$, значит свойства выше выполнятся не может и выполняется свойство (15).
        \end{note}

        \[ (14), (15) \Rightarrow \alpha \leq x_{n_0} \leq \beta \Rightarrow \alpha < \beta \]

        То-есть, мы действительно получили сечение.

    \end{itemize}
    Теперь можно применить теорему Дедекинда. По теореме Дедекинда:

    \[ \exists a \in R: \forall \alpha \in A, \forall \beta \in A': \alpha < a < \beta \eqno(16)\]


    Возьмём $\forall \epsilon > 0$, тогда:
    \[ (8) \implies \exists N \text{ такое, что выполнено (8)} \]

    $m = N + 1$

    Тогда, $(8) \Rightarrow \forall n > N + 1 $

    \[ |x_n - x_{N + 1}| < \epsilon \Leftrightarrow x_n \in (x_{N + 1} - \epsilon, x_{N + 1} + \epsilon) \eqno(17) \]

    Теперь, если посмотреть на соотношение (17),

    \[ (17) \Leftrightarrow x_n > x_{N + 1} - \epsilon \text{ и } x_n < x_{N + 1} + \epsilon \eqno(18) \]

    \begin{note}
        при $\forall n > N + 1$, выполнена правая счасть неравенства (17) $x_n > x_{N + 1} - \epsilon$.
    \end{note}

    Теперь рассмотрим (10) и (18).

    \[ (10), (18) \Rightarrow x_{N + 1} - \epsilon \in A \eqno(19) \]

    Теперь обратимся ко второму неравенству в соотношении (18).

    Получается, что правая часть неравенства $ x_n < x_{N + 1}$ принадлежит A', потому что если бы принадлежало A, должно было бы быть другое неравенство в другую сторону

    \[ (10), (18) \Rightarrow x_{N + 1} + \epsilon \in A' \eqno(20)\]

    Возьмём (19) $\Rightarrow x_{N + 1} - \epsilon$ как $\alpha$,

    а (20) $\Rightarrow x_{N + 1} - \epsilon$ как $\beta$,

    Тогда, применяем (16), получаем что:

    \[ (16), (19), (20) \Rightarrow x_{N + 1} - \epsilon \leq a \leq x_{N + 1} + \epsilon \eqno(21) \]

    Обратимся к соотношению (17)

    \[ (17): x_{N+1} < x_n < x_{N + 1} + \epsilon\]

    Получаем, что $a$ удовлетворяет этому неравенству и $x_n$ удовлетворяет этому неравенству (лежит на промежутке) при $\forall n > N + 1$.

    Поэтому, (21) и (17) $\Rightarrow $

    \[ |x_n - a| < 2 \epsilon = (x_{N + 1} + \epsilon) - (x_{N + 1} - \epsilon) \eqno(22) \]

    \begin{note}
        То-есть, если $x_n$ и a лежат на этом промежутке, то длина отрезка между a и $x_n$ меньше чем длина промежутка, на котором они лежат. Длина промежутка равна $2\epsilon$


    \end{note}

    Мы получили, что существует некоторое $a$ такое, что для любого n > N+1 выполняется неравенство (22). А это определение предела.

    По определению предела,

    \[ (22) \Rightarrow \lim_{n \to \infty} x_n = a \]

    Тем самым, достаточность в критерии доказана. 



\end{proof}

\section{Подпоследовательности}

\begin{definition}
    
    Пусть есть отображение $f: \N \to \R$ и нетождественное отображение $g: \N \to \N$. При этом выполняется: $\forall n < m: g(n) < g(m)$

    Тогда последователность отображений $f(g): \N \to \R$ --- подпоследовательность.
\end{definition}


\begin{note}
    $\{x_n\}_{n=1}^\infty$

    Берем $g(1) = n_1, g(2) = n_2, \ldots, g(k) = n_k$ и получаем подпоследовательность:

    $x_{n_1}, x_{n_2}, \ldots, x_{n_k}$
\end{note}

\begin{notation}
    Если эти номера определены, то последовательность обозначают как: $\{x_{n_k}\}_{k=1}^\infty$
\end{notation}


\begin{definition}
    Предел последовательности определяется как предел подпоследовательности по нижним индексам.

    Если есть такая последовательность, говорят что:

    $ A \in \overline{\R} $ является пределом, то есть $x_{n_k} \to A$, при $k \to \infty$, если $\forall \Omega (A) \; \exists K: \forall k > K: x_{n_k} \in \Omega(A)$
    
\end{definition}


\begin{theorem}
    Пусть $x_n \to A$, при $n \to \infty$, где $A \in \overline \R$ и пусть мы имеем любую подпоследовательность $\{x_{n_k}\}_{k=1}^\infty$, выбранную из этой последовательности. 

    Тогда $x_{n_k} \to A$, при $k \to \infty$.
\end{theorem}

\begin{proof}
    Возьмём любую окрестность A.

    \[ \forall \Omega (A) \Rightarrow \exists N: \forall n > N: x_n \in \Omega (A) \]

    Воспользуемся тем, что поледовательность $n_k$ строго возрастает:

    \[n_1 \geq 1, n_2 > n_1, n_2 \geq 2 \]

    Тогда по индукции:

    \[n_k \geq k \Rightarrow n_{k + 1} > n_k \geq k \rightarrow n_{k + 1} > k+1\]

    То есть, если мы выберем подпоследовательность, то $n_k$ будет больше или равно k. Начиная с какого-то индекса, будет строго больше.

    Возьмём $k = N$.

    Тогда, при $k > N: n_k \geq k > N$

    То есть, при $k > N: x_{n_k} \in \Omega (A)$

    \[\Rightarrow x_{n_k} \to A \text{, при $k \to \infty$} \]
\end{proof}


\begin{theorem}(Больцано-Вейерштрасса)


    Пусть имеется некоторая последовательность $\{x_n\}_{n=1}^\infty$, которая ограничена, т.е. $\forall n: a \leq x_n \leq b$.
    
    Тогда: $\exists \alpha \in [a, b] \text{ и } {x_{n_k}}_{k=1}^\infty$ такие, что: $x_{n_k} \to \alpha \text{ при } k \to \infty  $

    \begin{remark}
        Такое $\alpha$ может быть только одним, если последовательность ограниченна и имеет некоторый предел.
    \end{remark}

\end{theorem}

\begin{proof}
    определим последовательность промежутков.

    \[ I_1 = [a, b] \]

    \[ I'_2 = [a, \frac{a + b}{2}],  I''_2=[\frac{a+b}{2}, b]\]

    \begin{note}
        $\frac{a + b}{2}$ - это центр отрезка [a, b]
    \end{note}

    В последовательности $x_n$ имеется бесконечно ммного номеров (начиная с 1).

    Рассмотрим множество номеров в множестве n таких, что $x'_n \in I'_2$

    и n такие что $x_n \in I''_2$

    (Какое-то из них, или оба бесконечны.)

    Если бы первое и второе множество n выше было конечно, то мы получили бы что у нас есть конечное множество номеров n.

    А в силу соотношения 1 на всем промежутки $I_1$ лежит вся последовательность.

    поэтому, если бы и первое и второе множество было бы конечно, мы бы получили что рассматривам конечно множество номеров $x_n$, которые лежат на всем отрезке $I_1$, а на $I_1$ лежит вся последовательность.


    Пусть $I_2$ - тот из $I'_2$, $I''_2$, для которого $\exists$ бесконечно n таких что $x_n \in I_2$

    \begin{note}
        Это может быть либо $I'_1$, либо $I'_2$, либо $I''_2$ если оба удовлетворяем, то любой возьмем. Произвольно. Можно например всегда брать только $I'_2$, но по крайней мере для одного, таких номеров будет бесконечно много.
    \end{note}

    Имеется некоторое множество натуральных чисел, таких что $x_n$ принадлежит $I_2$

    Пусть $n_1$ - минимаьные n, такие что $x_n \in I_2$

    $I_2 = [a_2, b_2]$

    \begin{note}
        Снова рассмотрим середину, $\frac{a_2 + b_2}{2}$
    \end{note}

    \[I'_3 = [a_2, \frac{a_2+b_2}{2}] \]
    \[I''_3 = [\frac{a_2 + b_2}{2}, b_2] \]

    Нам известно, что множество тех n, таких что лежат на $I_2$, множество таких n - бесконечно.

    По крайней мере в одном из этих множеств тоже будет находится бесконечное множество номеров n.

    Пусть $I_3$ - тот из $I'_3$, $I''_3$, для которого $\exists$ бесконечно n таких что $x_n \in I_3$

    $n_2$ - минимальное n такое, что $x_n \in I_3$, и $n_2 > n_1$.

    \begin{note}
        Точка $x_n1$, может попасть на этот промежуток $I_3$, но посколько для этого промежутка существует бесконечно много n, таких что n пренадлежит промежутку $I_3$, то мы можем взять следующую, больше чем $n_1$, и называем её $n_2$


    \end{note}

    И так далее по индукции.
    Предположим, что мы уже выбрали промежутки
    \[ I_1 \supset I_2 \supset \dots \supset I_m \eqno(3') \]

    При этом мы всё время делим пополам.

    $k + 1 \leq m$

    длина $I_{k+1} = \frac{1}{2}$ длинны
    \[I_k = \frac{b - a}{2^k} \eqno(3) \]

    \[ n_1 < n_2 < \dots n_m < n_{m+1} \eqno(4)\]

    \[ x_{n_1} \in I_2, x_{n_2} \in I_2, \dots x_{n_{m-1}} \in I_m \eqno(5)\]

    Предположим, что по индукции такое построение уже произошло

    Пусть
    \[I_m = [a_m, b_m] \eqno(6) \]

    Индуктивное предположение (индуктивный шаг)

    Существует бесконечно много n, таких что
    \[ x_n \in I_m \eqno(7) \]

    Для двух и трёх мы это проделали.
    Предположим, что это проделано для n и будем выполнять индуктивный шаг.

    \[ I'_{m+1} = [a_m, \frac{a_m + b_m}{2}] \]

    \[ I''_{m+1} = [\frac{a_m + b_m}{2}, b_m] \]

    Мы снова взяли и разделили промежуток $[a_m, b_m]$ пополам.


    Рассмотрим множество номеров в множестве n таких, что $x'_n \in I'_{m+1}$

    и n такие что $x_n \in I''_{m+1}$

    (Хотя бы одно из них бесконечно, по той причине что объединение этих множеств это множество тех n таких что $x_n$ принаддлежит $I_m$,

    потому что вместе они дают на $I_m$, в силу предположения (7). Если бы и то и другое было бы конечно, то на множестве $I_m$ было бы конечно множество номеров таких что $x_n$ лежит на $I_m$, а по предположениб индукции их должно быть бесконечно.)

    Тогда по определению $I_{m+1}$ - тот ищ $I'_m, I''_m$, для которого $\exists$ бесконечно много n таких что $x_n \in I_{m+1}$

    Пускай $n_{m+1}$ - это наименьшее n такое что $x_{n_{m}} \in I_{m+1}$ и $n_{m+1} > n_m$

    \begin{note}
        Если элемент $x_{n_m}$ лежит на $I_{m+1}$, то мы вычеркиваем его и рассматриваем минимальный следующий (их бесконечно много).
    \end{note}

    И так мы получили в итоге этих рассуждений:

    \[ n_1 < n_2 < \dots < n_m < \dots \]

    \[x_{n_m} \in I_{m+1} \]

    \[ (3) \Rightarrow \text{ длина } I_m \to 0 \text{, при $m\to \infty$} \eqno(8) \]

    \begin{note}
        Получается, что это вложенные промежутки.

    \end{note}

    \[ (3') \text{и} (8) \]

    По теореме о вложенных пределах:

    \[ \exists! \alpha \text{ такое что } \alpha \in I_m \forall m \eqno(9) \]

    \[ (5) \Rightarrow x_{n_m} \in I_{m+1} \]

    Точка $\alpha$ лежит на этом промежутка и точка с номером $x_{n_m}$ лежит на этом же промежутке.

    \[ (5), (9) \Rightarrow |x_{n_m} - \alpha| \leq \frac{b - a}{2^m} \eqno(10) \]

    $ \forall \epsilon > 0 $

    $k: \frac{b - a}{2^k} < \epsilon$

    Возьмём $m > K$

    \[ (10), (11) \Rightarrow \text{при} m > K  \]

    выполнено \[ x_{n_m} - \alpha \to \alpha \text{при $m \to \infty$} \eqno(12)\]

    Таким образом мы доказали, что существует подпоследовательность у которой есть конечный предел.

    \[ a \in I_1 \], т.е. \[a \leq \alpha \leq e \]
\end{proof}

\section{Верхний и нижний предел последовательности}

\begin{definition}
    Пусть есть произвольная последовательность $x_n$.

    \[ {x_n}_{n=1}^\infty, x_n \in \R \]

    Если ${x_n}_{n=1}^\infty$ не ограничена сверху, то верхний предел $\overline{\lim_{n \ to \infty}} x = + \infty$, по определению.

    Если ${x_n}_{n=1}^\infty$ ограничена сверху, т.е. 
    
    \[ \exists M \text{ т.ч. } x_n \leq M \forall \eqno(1) \]

    \[ E_n = {a \in \R: a =  x_m, m \geq n} \]

    (множество всех значение последовательности $x_n$ начиная с множества n)

    \[ g_n = \sup E_n \]

    \[ (1) \Rightarrow E_n \text{ ограничена сверху} \Rightarrow  \]

    \[ g_n \leq M \forall n \eqno(2) \]

    Обратим внимание, что
    \[ E_{n+1} \subset E_n \Rightarrow g_{n+1} \leq g_n \eqno(3) \]
    Потому что может быть они совпадают, но мы рассматриваем элементов на 1 больше.

    \[ (3) \Rightarrow \exists \lim_{n \to \infty} g_n \geq - \infty \eqno(4) \]
    
    \[ (2) \Rightarrow \lim_{n \to \infty} g_n \leq M \eqno(5) \]

    \[ \overline{\lim_{n \ to \infty}} x_n = \lim_{n \to \infty} g_n \text{по определению} \]

    Если мы посмотрим на определение верхнего предела, видно, что верхний предел, в отличии от просто предела существует в нулевой последовательности.
    Т.к. последовательность либо ограничена сверху, либо не ограничена сверху.


    Если ${x_n}_{n=1}^\infty$ не ограничена снизу, то

    \[ \underline{\lim_{n \to \infty}} x_n \text{по определению равно} -\infty\]

    Если ${x_n}_{n=1}^\infty$ ограничена снизу, то-есть

    \[ \exists L, \text{ т.ч. } x_n \geq L \forall n \eqno(7)\]

    \[ h_n = \inf E_n \]

    \[ (7) \Rightarrow h_n > -\infty \]

    \[ h_{n+1} \geq h_n \eqno(8) \]

    $h_n$ - это монотонно возрастающая последовательность, а у любой такой последовательности есть предел. Может быть равный $+\infty$

    \[ (8) \Rightarrow \exists \lim_{n \to \infty} h_n \leq + \infty \]

    \[ \underline{\lim_{n \to \infty}} x_n \text{по определению равен} \lim_{n \to \infty} h_n \eqno(9) \]

    Таким образом,если мы рассматриваем любую последовательность $x_n$, то у неё существуют верхний и нижний предел.
\end{definition}

\section{Свойства верхних и нижних пределов}

\begin{enumerate}
    \item  \[ h_n = \inf E_n \leq \sup E_n = g_n \eqno(10)\]
    
    и последовательность $g_n$ и $h_n$ имеют пределы.

    Для всякого n спораведливо это неравенство (10)

    \[ (10) \Rightarrow \lim_{n \to \infty} h_n \leq lim_{n \to \infty} g_n \eqno(11) \]

    \[ (11): \underline{\lim_{n \to \infty}} x_n \eqno(12) \]

    \begin{note}
        В отличии от обычных пределов, верхние и нижние пределы существуют у любой последовательности.
    \end{note}

    \begin{theorem}
        Есть некоторая последовательность, тогда для того чтобы существовал предел 
        
        $\exists \lim_{n \to \infty} x_n = a \in \overline{\R}$

        необходимо и достаточно, чтобы 

        \[ \underline{\lim_{n \to \infty}} x_n = \overline{lim_{n \to \infty}} x_n = a \eqno(13) \]
    \end{theorem}



    \begin{proof}
    
        \begin{note}
            Здесь нужно рассмотреть все случаи, когда соотвествующие пределы и какой-то из них является символами + или - $\infty$, но мы рассмотрим только когда речь идет о когда оба предела это вещественные числа.
        \end{note}

        Предположим, что существует предел.

        Хотим проверить, что верхний предел равен нижнему пределу.

        \[ \forall \epsilon > 0 \exists N \text{т.ч.} \forall n > N \]

        \[ a - \epsilon < x_n < a + \epsilon \eqno(14) \]

        Посмотрим на определение $g_n$ и $h_n$.

        \[ (14) \Rightarrow \text{ при } n > N E_n \subset (a - \epsilon, a + \epsilon) \Rightarrow \]


        \[ \Rightarrow g_n \leq a + \epsilon, h_n \geq a - \epsilon \Rightarrow\]

        \[ \Rightarrow a - \epsilon \leq \underline{\lim} x_n = \overline{\lim x_n} \leq a + \epsilon \]

        \[ \underline{lim x_n} \geq a - \epsilon \Rightarrow\]

        \[ \Rightarrow 0 \leq \underline{\lim x_n} - \underline{\lim x_n} \leq 2 \epsilon \eqno(15) \]

        Получается, что некоторое не отрицательное число не превосходит $2 \epsilon$ при любом положительном $\epsilon$. Это может быть только тогда, когда это число равно 0.

        \[ (15) \Rightarrow \underline{\lim} x_n = \overline{\lim} x_n = \lim x_n \]

        И нижние и верхние пределы на самом деле равны a.

        % возможно здесь потеря причино-следственной связи из-за того что лектор куда-то перепрыгивал

        Тогда мы получаем следующие суждения 

        \[ g_n \to a, h_n \to a \]

        \[ g_n \geq a \forall n \]

        \[ h_n \leq a \forall n \]

        \[ \forall \epsilon > 0 \exists N_1 \text{т.ч.} a \leq g_n < a + \epsilon \text{при} n> N_1 \eqno(16) \]
        
        и 
        \[ \exists N_2 \text{ т.ч } a - \epsilon < h_n \leq a \text{при} n > N_2 \eqno(17) \]

        \[ N = \max (N_1, N_2) n > N \]

        \[ (16), (17) \Rightarrow a - \epsilon < \inf E_n \leq \sup E_n < a + \epsilon \eqno(18) \]

        \[ (18) \Rightarrow \forall m \geq n \text{выполнено} a - \epsilon < x_m < a + \epsilon \eqno(19) \]

        В частности, 
        \[a - \epsilon < x_n < a + \epsilon \eqno(20)\]

        \[(20): \exists \lim_{n \to \infty} x_n = a = \underline{\lim} x_n = \lim x \] 

        Теорема доказана.
    \end{proof}


\end{enumerate}