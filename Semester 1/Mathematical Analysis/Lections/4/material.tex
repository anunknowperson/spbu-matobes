
%\setcounter{chapter}{-1}




\lesson{4}{27.09.2023}{Продолжение}




\begin{properties} (Продолжение)
    \begin{enumerate}
        \item[5]  $x_n \neq c \; \forall n, x_n \to a, a \neq 0 => \frac{1}{x_n} \to \frac{1}{a}$
        \item[6] $\begin{cases}
                x_n \to a \text{из п. 5} \\
                y_n \to b
        \end{cases} \implies \frac{y_n}{x_n} \to \frac{a}{b}$
        \item[7] $x_n \leq y_n \forall n, x_n \to a, y_n b \implies a \leq b$
    \end{enumerate}
\end{properties}

\begin{proof} (5, 6, 7)
    \begin{enumerate}
        \item[5] 
        \begin{itemize}
            \item[I.] Возьмем $\epsilon_0 = \frac{|a|}{2} > 0$, тогда:
            
            $\exists N: \forall n > N: |x_n - a| < \epsilon_0 \implies |x_n| \geq |a| - |x_n - a| > |a| - \frac{|a|}{2} = \frac{|a|}{2}$
            \item[II.] $\forall \epsilon > 0: \exists N_1: \forall n > N_1: |x_n - a| < \epsilon$
        \end{itemize}

        $N_0 = max(N_1, N)$. При $n > N_0$ получаем:

        $|\frac{1}{x_n} - \frac{1}{a}| = |\frac{a - x_n}{x_n \cdot a}| = \frac{1}{|a|} \cdot \frac{1}{|x_n|} \cdot |x_n - a| \underset{(I), (II)}{<} \frac{1}{|a|} \cdot \frac{2}{|a|} \cdot \epsilon$
        \item[6] $\frac{y_n}{x_n} = y_n \cdot \frac{1}{x_n}$ --- далее по п. (4), (5).
        \item[7] Предположим, что $a > b$. Тогда $\epsilon_0 = \frac{a - b}{2} > 0 \implies \begin{cases}
            \exists N_1: \forall n > N_1: |x_n - a| < \epsilon_0 \\
            \exists N_2: \forall n > N_2: |y_n - b| < \epsilon_0 \\
        \end{cases} \implies \forall n > N_1 + N_2 + 1: y_n < \epsilon_0 + b = b + \frac{a - b}{2} = a - \frac{a - b}{2} = a - \epsilon_0 < x_n \implies y_n < x_n$ --- противоречие с условием.
    \end{enumerate}
\end{proof}


\begin{remark} (Различные промежутки)
    \begin{enumerate}
        \item $(a, b) = \{x \in R: a < x < b\}$ --- интервал (открытый промежуток)
        \item $[a, b] = \{x \in R: a \leq x \leq b\}$ --- замкнутный промежуток
        \item $[a, b) = \{x \in R: a \leq x < b\}$ --- полуоткрытый промежуток
        \item $(a, b] = \{x \in R: a < x \leq b\}$ --- полуоткрытый промежуток
    \end{enumerate}
\end{remark}


\section{Расширенное множество вещественных чисел}

\begin{definition}
    $\overline{R} = R \cup \{+\infty, -\infty\}$ --- расширенное множество вещественных чисел. При этом:

    $\forall x \in \R: x < +\infty, x > -\infty$
\end{definition}

\begin{remark} (Еще промежутки)
    \begin{enumerate}
        \item $(a, \infty) = \{x \in \R : x > a\}$
        \item $[a, \infty) = \{x \in \R : x \geq a\}$
        \item $(-\infty, a] = \{x \in \R : x < a\}$
        \item $(-\infty, a] = \{x \in \R : x \leq a\}$
    \end{enumerate}
\end{remark}


\begin{properties} (Продолжение свойств пределов)
    \begin{enumerate}
        \item[8] $\begin{cases}
            \forall n: x_n \leq y_n \leq z_n \\
            x_n \to a \\
            z_n \to a \\
        \end{cases} \implies y_n \to a$ --- теорема о двух миллиционерах
    \end{enumerate}
\end{properties}

\begin{proof}
    $\begin{cases}
        \forall \epsilon > 0: \exists N_1: \forall n > N_1: |x_n - a| < \epsilon \Leftrightarrow x \in (a - \epsilon, a + \epsilon) \\
        \forall \epsilon > 0: \exists N_2: \forall n > N_2: |z_n - a| < \epsilon \Leftrightarrow z \in (a - \epsilon, a + \epsilon)
    \end{cases} \implies \forall n > max(N_1, N_2):$

    $a - \epsilon < x_n \leq y_n \leq z_n < a + \epsilon \implies |y_n - a| < \epsilon$
    
\end{proof}

\section{Бесконечные пределы}

\begin{definition} (Бесконечные пределы)
    \begin{itemize}
        \item $\{x_n\}_{n=1}^{\infty}$, $x_n \to \infty$, $n \to \infty$

        $\lim_{n \to \infty} x_n = + \infty$, если:

        $\forall L \in \R \; \exists N: \forall n > N: x_n > L$

        \item $\{y_n\}_{n=1}^\infty$, $y_n \to -\infty$, $n \to \infty$

        $\lim_{n \to \infty} y_n = -\infty$, если:

        $\forall L \in \R: \; \exists N: \forall n > N: y_n < L$


    (возможно сокращение записи n-> далее.)
    \end{itemize}
\end{definition}

\section{Единообразная запись определения пределов}

\begin{definition}
    Окрестостью вещественного числа $a$ называется любой интервал $(a - \epsilon, a + \epsilon)$, где $\epsilon > 0$ (обозначается как $\omega(a)$).
\end{definition}

\begin{definition}
    Окрестность $+\infty: (L, +\infty), L \in \R$

    Окрестность $-\infty: (-\infty, L), L \in \R$
\end{definition}

\begin{definition}
    Пусть $\{x_n\}_{n=1}^{\infty}$, тогда $x_n \to a$, если:

    $\forall \omega(\alpha): \; \exists N: \forall n > N: x_n \in \omega(\alpha)$
\end{definition}


\begin{properties} (Доказать самостоятельно)
    
    Пусть $\{a_n\}_{n=1}^\infty, a \to +\infty$, $\{b_n\}_{n=1}^\infty, b \to -\infty$, тогда:
    \begin{enumerate}
        \item
        $\begin{aligned}
            c > 0: & c a_n \to +\infty, c b_n \to -\infty \\
            c < 0: & c a_n \to  -\infty, c b_n \to +\infty \\
        \end{aligned}$
        \item 
        $\begin{aligned}
            x_n \to x , x \in \R \cup \{+\infty\} &\implies a_n + x_n \to +\infty \\
            y_n \to y, y \in \R \cup \{-\infty\} &\implies b_n + y_n \to -\infty \\
        \end{aligned}$
        \item Возьмем $x_n, y_n$ из п. (2), тогда:
        
        $\begin{aligned}
            x > 0 & \implies a_n x_n \to + \infty, b_n x_n \to -\infty \\
            y < 0 & \implies a_n y_n \to -\infty, b_n y_n \to +\infty \\
        \end{aligned}$
        \item Если $\forall n: a_n \neq 0, b_n \neq 0$, тогда:
        
        $\begin{aligned}
            \frac{1}{a_n} & \to 0 \\
            \frac{1}{b_n} & \to 0 \\
        \end{aligned}$

        $\begin{aligned}
            \text{Если } x_n > 0, x_n \to 0 & \implies \frac{1}{x_n} \to +\infty \\
            \text{Если } x_n < 0, x_n \to 0 & \implies \frac{1}{x_n} \to -\infty \\
        \end{aligned}$
        \item $\forall n: x_n \leq y_n, x \to \alpha, y_n \to \beta; \alpha, \beta \in \overline \R \implies \alpha \leq \beta$
        \item $\begin{cases}
            \forall n: x_n \leq y_n \leq z_n \\
            x_n \to \alpha, \alpha \in \overline{\R}\\
            z_n \to \alpha \\
        \end{cases} \implies y_n \to \alpha$
    \end{enumerate}
\end{properties}

\begin{remark}
    
    $+\infty = +\infty$

    $-\infty = -\infty$

    $-\infty < +\infty$
\end{remark}

\begin{proof} (2, 6)
    \begin{enumerate}
        \item[2] 
        $\begin{cases}
            x \in \overline{\R} \implies \exists M: \forall n: |x_n - x| < M \implies x_n > x - M \\
            \forall L \in \overline{\R}: \exists N: \forall n > N: a_n > L
        \end{cases} \implies$
        
        $\implies a_n + x_n > L + x - M$, где правая часть --- любое число.

        
        
        \item[6] $\forall \epsilon > 0: \exists N_1: \forall n > N_1: x_n \in (\alpha - \epsilon, \alpha + \epsilon)$

        $\forall \epsilon > 0: \exists N_2: \forall n > N_2: z_n \in (\alpha - \epsilon, \alpha + \epsilon)$

        $N_0 = max(N_1, N_2)$

        $\forall n > N_0: x_n \leq y_n \leq z_n \implies y_n \in (\alpha - \epsilon, \alpha + \epsilon)$
    \end{enumerate}
\end{proof}

\section{Асимпотика}

\begin{definition} (О-большая и о-малая)
    \begin{enumerate}
        \item $x_n = o(1)$, если $x_n \to 0$
        \item $y_n = O(1)$, если $\exists C: \forall n: |y_n| \leq C$
        \item Пусть $\{a_n\}_{n=1}^\infty,\{b_n\}_{n=1}^\infty, \forall n: b_n \neq 0$, тогда:
        
        $a_n = o(b_n)$, если $\frac{a_n}{b_n} \to 0$

        \item Пусть есть $\{c_n\}, \{d_n\}$, тогда:
        
        $c_n = O(d_n)$, если $\exists C: |c_n| \leq C|d_n|$

    \end{enumerate}

\end{definition}

\begin{remark}
    Это не равенство в привычном смысле, следует читать его только слева направо.
\end{remark}

\section{Монотонные последовательности}

\begin{definition}(монотонные последовательности)
    \begin{itemize}
        \item $\{a_n\}_{n=1}^\infty $ монотонно возрастает, если $\forall n: a_n \leq a_{n+1}$ (возрастает строго если $a_n < a_{n+1}$)
        \item $\{b_n\}_{n=1}^\infty $ монотонно убывает, если $\forall n: b_n \leq b_{n+1}$
    \end{itemize}
\end{definition}

\begin{remark}
    Говорят, что поледовательнотсть $c_n$ монотонна, если она либо монотонно возрастает, либо монотонно убывает.
\end{remark}

\begin{theorem} (Теорема о пределе монотонной последовательности)
    \begin{itemize}
        \item Пусть есть последовательность $\{c_n\}_{n=1}^\infty$, тогда $\exists \lim_{n \to \infty} c_n \in \overline{\R}$.

        \item Для того, чтобы монотонно возрастающая последовательность имела конечный предел, необходимо и достаточно, чтобы последовательность была ограничена сверху.

        \item Для того, чтобы монотонно убывающая последовательность имела конечный предел необходимо и достаточно, чтобы последовательность была ограничена снизу.
    \end{itemize}
    При этом справелдивы неравенства:
    \begin{itemize}
        \item $\forall m: c_m \leq \lim_{n \to \infty} c_n$ --- если последовательность возрастает. (или < если строго возрастает)
        \item $\forall m: c_m \geq \lim_{n \to \infty} c_n$ --- если последовательность убывает.
    \end{itemize}

\end{theorem}

\begin{proof}
    \begin{enumerate}
        \item Предположим, что проследовательность $c_n$ не ограничена сверху, тогда:

        $\forall L \in \R: \exists N: c_N > L$


        $\forall n > N: c_n \geq c_{n-1} \geq c_{n-2} \geq ... \geq c_N + 1 \geq c_N > L$, значит $c_n > L$

        Значит по определению предела: $\lim c_n = +\infty$
        
        
        \item Предположим теперь, что последовательность $c_n$ возрастает и ограничена сверху, тогда:

        $\begin{cases}
            c_n \leq c_{n+1} \\
            \exists M: \forall n: c_n \leq M \\
        \end{cases}$

        Пусть $E = \{\alpha \in \R: \exists n \in \N: \alpha = c_n\}$ --- множество из всех элементов последовательности $c_n$.
        
        Значит $E$ --- ограничено сверху. Положим $C = \sup E$, тогда имеем $\forall n: c_n \leq C$

        $\forall \epsilon > 0: C - \epsilon$ --- не верхняя граница, значит $\exists N: c_N > C - \epsilon \implies \forall n > N: c_n \geq c_{n-1} \geq \ldots \geq c_N > C - \epsilon \implies C - \epsilon < c_n \leq C < C + \epsilon \implies |c_n - C| < \epsilon \implies \lim_{n \to \infty} c_n = C$
        
        $\newline$
        
        В обратную сторону: если $\exists \lim_{n \to \infty} c_n = C \in \R \implies \exists M: \forall n: |c_n - C| < M \implies \forall n: c_n \leq C + M$
        \item Доказательство для убывающей последовательности аналогично.
    \end{enumerate}
\end{proof}

\begin{theorem} (Теорема о вложенных промежутках)
    
    Пусть $\forall n: [a_n, b_n] \supset [a_{n + 1}, b_{n + 1}]$ и $b_n - a_n \underset{n \to \infty}{\to} 0$. 
    
    Тогда $\exists! c: \forall n: c \in [a_n, b_n]$
\end{theorem}

\begin{proof}
    \begin{enumerate}
        

        \item существование
        
        имеем неравенства:
        
        $\forall n: \begin{cases}
            a_n \leq a_{n + 1} \\
            b_n \geq b_{n + 1} \\
            a_n < b_n \\ 
        \end{cases} \implies a_n < b_1, b_n > a_1$ 
        
        Тогда в силу возрастания $a_n$ и убывания $b_n$ по предыдущей теореоме $\exists a = \lim_{n \to \infty} a_n$ и $\exists b = \lim_{n \to \infty} b_n$

        По свойству перехода к пределу в неравенствах: $a_n < b_n \implies a \leq b$
        
        Имеем $\begin{cases}
            \forall n: a_n \geq a \\
            \forall n: b \leq b_n \\
        \end{cases} \implies \forall n: b - a \leq b_n - a_n \implies $
        
        $\implies 0 \leq \lim_{n \to \infty} (b - a) \leq \lim_{n \to \infty} (b_n - a_n) = 0$ --- в силу условия.

        $\newline$
        
        Значит $b - a = 0 \implies a = b \overset{def}{=} c$

        Имеем $a_n \leq c \leq b_n$, т.е. $c \in [a_n, b_n]$
        \item Единственность
        
        Если бы $\exists c_0 \in [a_n, b_n]$, то $|c_0 - c| \leq b_n - a_n \implies |c_0 - c| < \lim_{n \to \infty} (b_n - a_n) = 0 \implies c_0 = c$
    \end{enumerate}
\end{proof}

\begin{remark}
    
    Условие замкнутости промежутков существенно:

    Имеем $(0, \frac{1}{n + 1}] \supset (0, \frac{1}{n}]$, $\frac{1}{n} - 0 \underset{n \to \infty}{\to} 0$

    Но $\bigcap_{n=1}^{\infty} (0, \frac{1}{n}] = \varnothing$
    
\end{remark}

\section{Число $e$}

\begin{theorem}
    
    Пусть $x_n = (1 + \frac{1}{n})^n$ и $y_n = (1 + \frac{1}{n})^{n+1}$

    Тогда $\forall n: x_n < y_n$ и $x_n \to e,y_n \to e, 2 < e < 3$
\end{theorem}

\begin{proof}
    Рассмотрим: 
    \[ \frac{y_{n-1}}{y_n} = \frac{(\frac{n}{n - 1})^n}{(\frac{n + 1}{n})^{n+1}} = (\frac{n}{n + 1})^{n+1} \cdot (\frac{n}{n - 1})^{n} = (\frac{n}{n + 1})^{n} \cdot (\frac{n}{n + 1})^{n} \cdot (\frac{n}{n - 1})^{n} =\]

    \[= \frac{n}{n + 1} \cdot (\frac{n^2}{n^2 - 1})^n = \frac{n}{n + 1} \cdot (\frac{n^2 - 1 + 1}{n^2 - 1})^n = \frac{n}{n + 1} \cdot (1 + \frac{1}{n^2 - 1})^n\]

    Возьмем за $x = \frac{1}{n^2 - 1}$, тогда по неравенству Бернулли:
    \[\frac{n}{n + 1} \cdot (1 + \frac{1}{n^2 - 1})^n > \frac{1}{n + 1} \cdot (1 + \frac{n}{n^2 - 1}) = \frac{1}{n + 1} \cdot \frac{n^2 -1 + n}{n^2 - 1} =\]

    \[\frac{n^3 + n^2 - n}{n^3 + n^2 - n - 1} > 1\] $\implies y_{n} < y_{n-1} \implies y_n$ --- строго монотонно убывающая.

    $\newline$

    Теперь рассмотрим $x_n:$ (считаем, что $n \geq 3$)

    \[x_n = (1 + \frac{1}{n}^n) = \sum_{k=0}^{n}C_{n}^{k}(\frac{1}{n})^k = 1 + n \cdot \frac{1}{n} + \sum_{k=2}^{n}C_{n}^{k}\frac{1}{n^k} = \] 
    
    \[= 2 + \sum_{k=2}^{n}\frac{n!}{k!(n-k)!}\frac{1}{n^k} = 2 + \sum_{k=2}^{n}\frac{(n-k+1)\cdot \ldots \cdot n}{k! \cdot n^k} = \]
    
    \[= 2 + \sum_{k=2}^{n}\frac{1}{k!} \cdot (1 - \frac{k-1}{n}) \cdot (1 - \frac{k-2}{n}) \cdot (1 - \frac{1}{n})\]

    (Продолжение на следующей лекции)
\end{proof}