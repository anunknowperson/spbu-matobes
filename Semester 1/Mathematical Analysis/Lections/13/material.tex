

\lesson{13}{30.11.2023}{Экстремум и производная. Правило Лопиталя.}


\section{Достаточное условие локального экстремума с второй производной}
\begin{theorem}
	$f: (a,b) \to \mathbb{R}$\\
	$\forall x \in (a,b), \exists f'(x)$ \\
	$x_0 \in (a,b), \exists f''(x_0)$ \\
	$f'(x_0)=0, f''(x_0)>0 \implies x_0$ $-$ строгий локальный минимум $f$
\end{theorem}
\begin{theorem}
	$g: (a,b)\to \mathbb{R}$ \\
	$\exists g'(x)$ \\
	$\exists g''(x_0)$ \\
	$g'(x_0)=0, g''(x_0)<0 \implies x_0$ $-$ строгий локальный максимум $g$
\end{theorem}
\begin{replacementproof}
	\textbf{Формула Тейлора с остатком в форме Пиано}\\
	$f(x)=f(x_0)+f'(x_0)(x-x_0)+\displaystyle\frac{1}{2}f''(x_0)(x-x_0)^2+r(x)$ (1)\\
	$\displaystyle\frac{r(x)}{(x-x_0)^2} \underset{x\to x_0}{\to} 0$ (2)\\
	$f'(x_0)=0: \quad (1) \implies f(x)=f(x_0)+\displaystyle\frac{1}{2}f''(x_0)(x-x_0)^2+r(x)$ (3)\\
	$\varepsilon=\displaystyle\frac{1}{4}f''(x_0)$\\
	$(2) \implies \exists \omega(x_0): \forall x \in \omega(x_0):$
	\[
	|\displaystyle\frac{r(x)}{(x-x_0)^2}|<\varepsilon=\displaystyle\frac{1}{4}f''(x) \quad (4)
	\]
	$x\neq x_0, x \in \omega$\\
	(3)(4) $\implies f(x)\ge f(x_0)+\displaystyle\frac{1}{2}f''(x_0)(x-x_0)^2-|r(x)|>f(x_0)+\displaystyle\frac{1}{2}f''(x_0)(x-x_0)^2-\\ -\displaystyle\frac{1}{4}f''(x_0)(x-x_0)^2=f(x_0)+\displaystyle\frac{1}{4}f''(x_0)(x-x_0)^2>f(x_0)$
\end{replacementproof}
\subsection{Теорема о достаточном условии локального экстренума четной производной}
\begin{theorem}
	$f: (a,b)\to \mathbb{R}$ \\
	$n\ge 2: \quad \forall x \in (a,b), \exists f'(x), f''(x), \ldots,f^{(2n-1)}(x)$ \\
	$x_0 \in (a,b), \exists f^{(2n)}(x_0)$ \\
	$f'(x_0)=0, f''(x_0)=0, \ldots, f^{(2n-1)}(x_0)=0$ \\
	$f^{(2n)}(x_0)\neq 0$\\
	если $f^{(2n)}(x_0)>0$, то $x_0$ $-$ строгий локальный минимум\\
	если $f^{(2n)}(x_0)<0$, то $x_0$ $-$ строгий локальный максимум

\end{theorem}
\begin{replacementproof}
	$f(x)=f(x_0)+f'(x_0)(x-x_0)+\displaystyle\frac{1}{2}f''(x_0)(x-x_0)^2+\displaystyle\frac{1}{3!}f'''(x-x_0)^{3}+\\ +\ldots+\displaystyle\frac{1}{(2n)!}f^{(2n)}(x_0)(x-x_0)^{2n}+r(x)$ (5)
	\[
	\displaystyle\frac{r(x)}{(x-x_0)^{2n}} \underset{x\to x_0}{\to} 0 \quad (6)
	\] 
	$(5)\implies f(x)=f(x_0)+\displaystyle\frac{1}{(2n)!}f^{(2n)}(x_0)(x-x_0)^{2n}+r(x)$ (7)\\
	$\varepsilon = \displaystyle\frac{1}{2}*\displaystyle\frac{1}{(2n)!}*f^{(2n)}(x_0)$ \\
	(6) $\implies \exists \omega(x_0): \forall x \in \omega(x_0):$ 
	\[
	|\displaystyle\frac{r(x)}{(x-x_0)^{2n}}|<\varepsilon \quad (8)
	\] 
	$x \in \omega(x_0), x \neq x_0$ \\
	(7)(8) $\implies f(x)\ge f(x_0)+\displaystyle\frac{1}{(2n)!}f^{(2n)}(x_0)(x-x_0)^{2n}-|r(x)|>\\ >f(x_0)+\displaystyle\frac{1}{(2m)!}f^{(2n)}(x_0)(x-x_0)^{2n}-\displaystyle\frac{1}{2}*\displaystyle\frac{1}{(2n)!}f^{(2n)}(x_0)(x-x_0)^{2n}=\\ =f(x_0)+\displaystyle\frac{1}{2}*\displaystyle\frac{1}{(2n)!}f^{(2n)}(x_0)(x-x_0)^{2n}>f(x_0)$
\end{replacementproof}


\subsection{Достаточное условие отсутствия локального экстренума с нечетной производной}
\begin{theorem}
	$f: (a,b) \to \mathbb{R}$ \\
	Предположим $x_0 \in (a,b)$ \\
	$n\ge 1: \quad \forall x \in (a,b), \exists f'(x),f''(x),\ldots,f^{(2n)}(x); \quad \exists f^{(2n+1)}(x_0)$ \\
	$f'(x_0)=0,f''(x_0)=0,\ldots,f^{(2n)}(x_0)=0$ \\
	$f^{(2n+1)}(x_0)\neq 0$ \\
	Тогда $x_0$ $-$ не является точкой локального экстренума
\end{theorem}
\begin{replacementproof}
	$f(x)=f(x_0)+\displaystyle\frac{1}{(2n+1)!}f^{(2n+1)}(x_0)(x-x_0)^{2n+1}+r(x)$ \\
	$|\displaystyle\frac{r(x)}{(x-x_0)^{2n+1}}|\underset{x\to x_0}{\to} 0$ \\
	Возьмем окрестность $x_0$ $-$ $\omega(x_0): |\displaystyle\frac{r(x)}{(x-x_0)^{2n+1}}|<\displaystyle\frac{1}{2}*\displaystyle\frac{1}{(2n+1)!}|f^{(2n+1)}(x_0)|$ \\
	$x>x_0:\\ \quad f(x)>f(x_0)+\displaystyle\frac{1}{(2n+1)!}f^{(2n+1)}(x_0)(x-x_0)^{2n+1}-\displaystyle\frac{1}{2}\displaystyle\frac{1}{(2n+1)!}f^{(2n+1)}(x_0)(x-x_0)^{2n+1}>f(x_0)$\\
	$x<x_0:\\ f(x)<f(x_0)+\displaystyle\frac{1}{(2n+1)!}f^{(2n+1)}(x_0)(x-x_0)^{2n+1}+\displaystyle\frac{1}{2}\displaystyle\frac{1}{(2n+1)!}f^{(2n+1)}(x_0)|(x-x_0)^{2n+1}|=\\=f(x_0)+\displaystyle\frac{1}{2}\displaystyle\frac{1}{(2n+1)!}f^{(2n+1)}(x_0)(x-x_0)^{2n+1}<f(x_0)$
\end{replacementproof}


\section{Правило Бернулли$-$Лопиталя}
\begin{theorem}\textbf{Номер 1}\\
	Пусть $f,g: (a,b) \to \mathbb{R}$ \\
	Пусть $f(x) \neq 0, \forall x \in (a,b)$ \\
	$\forall x \in (a,b), \exists f'(x), \exists g'(x)$ \\
	Предположим $f'(x)\neq 0, \forall x \in (a,b)$ \\
	$f(x) \underset{x\to a+0}{\to} 0, g(x)\underset{x\to a+0}{\to} 0$ \\
	$\exists \displaystyle\lim_{x \to a+0} \displaystyle\frac{g'(x)}{f'(x)}=A \in \overline{\mathbb{R}}$ (1)\\
	$\implies \displaystyle\frac{g(x)}{f(x)} \underset{x\to a+0}{\to} A$ (2)
\end{theorem}
\begin{replacementproof}
	$f(a)=^{def}0, g(x)=^{def}0$ \\
	$f,g \in C([a,b))$ \\
	$b>x>a: \quad [a,x]$ по теореме Коши $\implies \exists c \in (a,x):$ 
	\[
	\displaystyle\frac{g(x)-g(a)}{f(x)-f(a)}=\displaystyle\frac{g'(c)}{f'(c)} \quad (3)
	\] 
	(3) $\implies \displaystyle\frac{g(x)}{f(x)}=\displaystyle\frac{g'(c)}{f'(c)}$ (4)\\
	$\forall \omega(A), \exists \delta >0: \forall y \in (a,a+\delta):$ 
	\[
	(1) \implies \displaystyle\frac{g'(y)}{f'(y)} \in \omega(A) \quad (5)
	\] 
	$x \in (a,a+\delta); \quad c \in (a,x)\implies c \in (a,a+\delta)$ \\
	(5) $\implies \displaystyle\frac{g'(c)}{f'(c)} \in \omega(A)$ (6)\\
	$(4)(6) \implies \displaystyle\frac{g(x)}{f(x)} \in \omega(A) \implies (2)$
\end{replacementproof}
\begin{theorem}\textbf{Номер 1'}\\
	Пусть $f,g: (a,b) \to \mathbb{R}$ \\
	Пусть $f(x) \neq 0, \forall x \in (a,b)$ \\
	$\forall x \in (a,b), \exists f'(x), \exists g'(x)$ \\
	Предположим $f'(x)\neq 0, \forall x \in (a,b)$ \\
	$f(x)\underset{x\to b-0}{\to} 0, g(x) \underset{x\to b-0}{\to} 0$ \\
	$\exists \displaystyle\lim_{x \to b-0} \displaystyle\frac{g'(x)}{f'(x)}=A \in \overline{\mathbb{R}}$ \\
	$\implies \displaystyle\frac{g(x)}{f(x)}\underset{x\to b-0}{\to} A$
\end{theorem}
\begin{replacementproof}
	Аналогично
\end{replacementproof}
\begin{theorem}\textbf{Номер 2}\\
	$f,g: (a,+\infty) \to \mathbb{R}$ \\
	$f(x)\neq 0, \forall x \in (a,+\infty)$ \\
	$f(x)\underset{x\to +\infty}{\to} +\infty$ (7)\\
	$\forall x \in (a, +\infty), \exists f'(x),g'(x)$ \\
	 Пусть $f'(x)\neq 0, \forall x \in (a,+\infty)$ \\
	 $\displaystyle\frac{g'(x)}{f'(x)}\underset{x\to +\infty}{\to}A$ (8)\\
	 $\implies \displaystyle\frac{g(x)}{f(x)}\underset{x\to +\infty}{\to} A$ (9)
\end{theorem}
\begin{replacementproof}
	$\forall \varepsilon >0$\\
	(8) $\implies \exists L_1: \forall x>L_1: \displaystyle\frac{g'(x)}{f'(x)}\in (A-\varepsilon,A+\varepsilon)$ (10)\\
	Возьмем $x>L_1, x>x_0$\\
	По теореме Коши $\exists c \in (x_0,x): \displaystyle\frac{g(x)-g(x_0)}{f(x)-f(x_0)}=\displaystyle\frac{g'(c)}{f'(c)}$ (11)\\
	$L_1<x>x_0 \implies c>L_1$\\
	$\varepsilon < \displaystyle\frac{1}{2}$
	$\displaystyle\frac{g'(c)}{f'(c)}\in(A-\varepsilon,A+\varepsilon)$ (12)\\
	$\displaystyle\frac{g(x)-g(x_0)}{f(x)-f(x_0)}=\displaystyle\frac{\displaystyle\frac{g(x)}{f(x)}-\displaystyle\frac{g(x_0)}{f(x)}}{1-\displaystyle\frac{f(x_0)}{f(x)}}$ (13)\\
	$L_2\ge L_1$, при $x>L_2$\\
	$|\displaystyle\frac{g(x_0)}{f(x)}|<\varepsilon, |\displaystyle\frac{f(x_0)}{f(x)}|<\varepsilon$ (14)\\
	(14) $\implies -2\varepsilon=-\displaystyle\frac{\varepsilon}{1-\displaystyle\frac{1}{2}}<\displaystyle\frac{\displaystyle\frac{g(x_0)}{f(x)}}{1-\displaystyle\frac{f(x_0)}{f(x)}}<\displaystyle\frac{\varepsilon}{1-\displaystyle\frac{1}{2}}=2\varepsilon$ (15)\\
	(11)(12)(13) $\implies x>L_2: A-3\varepsilon<\displaystyle\frac{\displaystyle\frac{g(x)}{f(x)}}{1-\displaystyle\frac{f(x_0)}{f(x)}}<A+\varepsilon+\displaystyle\frac{\displaystyle\frac{g(x_0)}{g(x)}}{1-\displaystyle\frac{f(x_0)}{f(x)}}<A+3\varepsilon$ (16)\\
	(14)(16) $\implies \displaystyle\frac{g(x)}{f(x)}<(A+3\varepsilon)(1-\displaystyle\frac{f(x_0)}{f(x)})<(A+3\varepsilon)(1+\varepsilon)=A+(A+3)\varepsilon+3\varepsilon^2$ (17)\\
	$\displaystyle\frac{g(x)}{f(x)}>(A-3\varepsilon)(1-\displaystyle\frac{f(x_0)}{f(x)})>(A-3\varepsilon)(1-\varepsilon)=A-(A+3)\varepsilon+3\varepsilon ^2$ (18)\\
	(17)(18) $\implies (9)$
\end{replacementproof}
\textbf{Следствие:} $x>1, g(x)=\ln {x}, f(x)=x^{r}, r>0$ \\
$g'(x)=\displaystyle\frac{1}{x}, f'(x)=rx^{r-1}$ \\
$\displaystyle\frac{g'(x)}{f'(x)}=\displaystyle\frac{\displaystyle\frac{1}{x}}{rx^{r-1}}=\displaystyle\frac{1}{rx^{r}} \underset{x\to +\infty}{\to} 0 \implies \displaystyle\frac{\ln {x}}{x^{r}} \underset{x\to +\infty}{\to} 0$
\begin{theorem}\textbf{Номер 3}\\
	$f,g: (a,b) \to \mathbb{R}$ \\
	$x_0\in(a,b)$ \\
	$f(x)\neq 0$, если $x\neq x_0$\\
	$n\ge 2: \quad \forall x \in (a,b), \exists f'(x),\ldots,f^{(n)}(x); \quad \exists g'(x),\ldots,g^{(n-1)}(x); \quad \exists f^{(n)}(x_0), g^{(n)}(x_0)$ \\
	$f(x_0)=f'(x_0)=\ldots=f^{(n-1)}(x_0)=0$ \\
	$g(x_0)=g'(x_0)=\ldots=g^{(n-1)}(x_0)=0$ \\
	Пусть $f^{(n)}\neq 0$ \\
	\[
	\displaystyle\frac{g(x)}{f(x)} \underset{x\to x_0}{\to} \displaystyle\frac{g^{(n)}(x_0)}{f^{(n)}(x_0)} \quad (19)
	\] 
\end{theorem}
\begin{replacementproof}
	 Теорема Тейлора с остатком в форме Пиано $\implies \\ \displaystyle\frac{g(x)}{f(x)}=\displaystyle\frac{\displaystyle\frac{g^{(n)}(x_0)}{n!}(x-x_0)^{n}+r_1(x)}{\displaystyle\frac{f^{(n)}(x_0)}{n!}(x-x_0)^{n}+r_2(x)}=\displaystyle\frac{g^{(n)}(x_0)+n!\displaystyle\frac{r_1(x)}{(x-x_0)^{n}}}{f^{(n)}(x_0)+n!\displaystyle\frac{r_2(x)}{(x-x_0)^{n}}} \underset{x\to x_0}{\to} \displaystyle\frac{g^{(n)}(x_0)}{f^{(n)}(x_0)}$
\end{replacementproof}
\end{document}
