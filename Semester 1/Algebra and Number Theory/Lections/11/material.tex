\DeclareRobustCommand{\divby}{%
  \mathrel{\vbox{\baselineskip.65ex\lineskiplimit0pt\hbox{.}\hbox{.}\hbox{.}}}%
}

% разобрать: 
%1 формулу Лагранжа (proof) 
%2 табилцу в примере в начале
%3 пример после Лагранжа допиши с itemize
\lesson{11}{17.11.2023}{}

\section{Интерполя}

%\begin{remark} По точкам можно найти многочлен
%  \begin{tabular}{c|c|c|c}
%    $x$ & $x_1$ & $x_2$ & $\ldots$ & $x_n$ \\
%    \hline
%    $F(x)$ & $y_1$ & $y_2$ & $\ldots$ & $y_n$ \\
%  \end{tabular}
%\end{remark}

\begin{theorem} (Интерполяционная формула Лагранжа)
  
  Пусть $K$ --- поле. $\forall x_1, \ldots, x_n \in K$ и $\forall y_1, \ldots, y_n \in K$, тогда : 
  $$\exists! F \in K[x]: \forall i: F(x_i) = y_i$$
  
  Многочлен можно найти по формуле:

  $$F(x) = L_1(x)y_1 + L_2(x)y_2 + \ldots + L_n(x)y_n$$

  $$\text{ где } L_i(x) =\frac{(x - x_1) \cdot \ldots \cdot (x - x_{i-1}) \cdot (x - x_{i+1}) \cdot \ldots \cdot (x - x_n)}{(x_i - x_1) \cdot \ldots \cdot (x_i - x_{i-1}) \cdot (x_i - x_{i+1}) \cdot \ldots \cdot (x_i - x_n)}$$
\end{theorem}

\begin{proof}
 
  $\newline$

  \begin{enumerate}
    \item Существование. Проверим, что многочлен, заданный формулой, подходит:
    
    $\deg L_i = n - 1 \implies \begin{cases}
      \deg (L_i(x) y_i) = n - 1\\
      L_i (x) y_i = 0
    \end{cases} \implies \deg F \leq \max \{\deg L_i\} = n - 1$

    $\forall i: L_i(x_i) = 1$ и $\forall i \neq j: L_i(x_j) = 0$

    $F(x_i) = L_1(x_i)y_1 + \ldots + L_i(x_i)y_i + \ldots + L_n(x_i)y_n = 0 \cdot y_1 + \ldots + 1 \cdot y_i + \ldots + 0 \cdot y_n = y_i$
    
    \item Единственность. Пусть $F(x), G(x): \forall i: F(x_i) = y_i \land G(x_i) = y_i$ тогда:
    
    $\deg F \neq n - 1, \deg G \neq n - 1$

    Пусть $H(x) =F(x) - G(x) \implies \deg H \neq n - 1$

    $H(x_i) = y_i - y_i = 0$ --- у $H$ есть $n$ корней, значит $H = 0 \implies F = G$
  \end{enumerate}
\end{proof}

\begin{eg}
  \begin{tabular}{c|c|c|c}
    $x$ & $1$ & 2 & 3 \\
    \hline
    $F(x)$ & 1 & 4 & 9 \\
  \end{tabular}
  
  $\newline$

  $L_1(x) = \frac{(x-2)(x-3)}{(1-2)(1-3)} = \frac{1}{2}x^2 - \frac{5}{2}x + 3$

  $L_2(x) = \frac{(x-1)(x-3)}{(2-1)(2-3)} = -x^2 + 4x - 3$

  $L_3(x) = \frac{(x-1)(x-2)}{(3-2)(3-2)} = \frac{1}{2}x^2 - \frac{3}{2}x + 1$

  $F(x) = (\frac{1}{2}x^2 - \frac{5}{2}x + 3) \cdot 1 + (-x^2 + 4x - 3) \cdot 4 + (\frac{1}{2}x^2 - \frac{3}{2}x + 1) \cdot 9 = x^2$
\end{eg}

\section{Метод интерполяции Ньютона}

\begin{algoritm} (Ньютона)

  Для данных различных $x_1, x_2, \ldots, x_n$ и данных $y_1, y_2, \ldots, y_n$ требуется построить многочлен $F(x)$ (он уже существует и единственен) такой, что:
  $$\forall i: F(x_i) = y_i, \deg F \leq n - 1$$

  Построим последовательно многочлены $f_1, f_2, \ldots, f_n$ так, что: $\deg f_k(x) \leq k - 1$ и $f_k(x_1) = y_1, \ldots, f_k(x_k) = y_k$

  На n-ом шагу подойдет многочлен $f_n(x) = F(x)$

  $\newline$

  В начале возьмем $f_1(x) = y_1$. Многочлен $f_k(x)$ определим по формуле:

  $$f_k(x) = f_{k-1}(x) + A_{k-1} \cdot g_{k-1}(x) \text{, где } g_{k-1}$$

  $$g_{k-1}(x) = (x - x_1) \cdot \ldots \cdot (x - x_{k-1}), \;\; A_{k-1} = \frac{y_k-f_{k-1}(x_k)}{g_{k-1}(x_k)}$$
\end{algoritm}


\begin{eg}
  \begin{tabular}{c|c|c|c}
    $x$ & $1$ & 2 & 3 \\
    \hline
    $F(x)$ & 1 & 4 & 9 \\
  \end{tabular}
  
  $\newline$
  \begin{itemize}
    \item $f_1(x) = 1$

    \item $f_2(X) = 1 + A(x-1)$. 
    
    Найдем А: $x = 2 \implies 4 = 1 + A(2 - 1) \implies A = 3$. Значит, $f_2(x) = 3x - 2$

    \item $f_3(x) = (3x - 2) + A(x-1)(x-2)$

    Найдем А: $x = 3 \implies 9 = (3 \cdot 3 - 2) + A(3 - 1)(3 - 2) \implies A = 1 \implies $
    
    $\implies f_3(x) = (3x - 2) + 1\cdot (x-1)(x-2) = x^2$. Значит, $F(x) = x^2$
    \end{itemize}
\end{eg}

\begin{theorem}
  Метод интерполяции Ньютона работает корректно определен и результат его применения --- нужный многочлен.
\end{theorem}

\begin{proof}
  Имеем $g_{k-1} = (x_k - x_1) \cdot \ldots \cdot (x_k - x_{k-1}) \neq 0 \implies A_k$ --- определено.

  Докажем по индукции, что $\deg f_k \leq k-1$:

  База: $\deg f_1 = \left[\begin{gathered}
    0 \\
    -\infty 
  \end{gathered}\right. \leq 1 - 1$

  Переход $k - 1 \to k$: $\deg f_k \leq \max \{\underset{\leq k - 2}{\deg f_{k-1}}, \underset{=k-1}{\deg (A_k \cdot g_{k-1})}\} \leq k - 1$

  $\newline$

  Докажем по индукции, что $f_k(x_1) = y_1, \ldots, f_k(x_n) = y_n$:
\end{proof}

\section{Делимость в области целостности} (По учебнику Кострикина)

\begin{definition}
  Пусть $A$ --- область целостности, $a, b \in A$. a делится на b, если $\exists c \in A: a = bc$. Обозначается как $a \divby b$
\end{definition}


\begin{properties} (доказательство в качестве упражнения)
  
  \begin{enumerate}
    \item $a, b \divby c \implies a + b \divby c, a - b \divby c$
    \item $a \divby b \implies \forall k \in A: ak \divby b$
    \item $a \divby b, b \divby c \implies a \divby c$
  \end{enumerate}
\end{properties}

\begin{definition}
  Пусть $A$ --- область целостности. Элементы $a, b$ называются ассоциированными, если $a \divby b, b \divby a$
\end{definition}

\begin{eg}
  $\newline$

  \begin{enumerate}
    \item в $\Z:$ a ассоциировано с $a$ и с $-a$
    \item в $\R[x]: P(x), Q(x)$ ассоциированы, если $P(x) = c \cdot Q(x), c \neq 0$
  \end{enumerate}
\end{eg}

\begin{properties}
  Пусть $A$ --- область целостности с единицей. Тогда:
  \begin{enumerate}
    \item a и b --- ассоциированы $\Leftrightarrow \exists u: $ u - обратим и $a = b \cdot u$
    \item $\begin{cases}
      a \divby b \\
      a, a_1  \text{ --- ассоциированы} \\
      b, b_1  \text{ --- ассоциированы} \\
    \end{cases} \implies a_1 \divby b_1$
  \end{enumerate}
\end{properties}


\begin{proof}
  
  $\newline$
  
  \begin{enumerate}
    \item \begin{itemize}
      \item[$\implies:$] $a = bc, b = ad \implies a = b \cdot c = (ad) \cdot c = a(dc) \implies 1 = dc \implies$ c --- обратим 
      \item[$\Leftarrow:$] $a = bu \implies a \divby b$
      
      $\exists u^{-1}$, т.к. u обратим

      $au^{-1} = b \cdot u \cdot u^{-1} = b \cdot 1 = b \implies b \divby a$
    \end{itemize}
    \item $a = bc, a = u\cdot a_1$, где u --- обратим и $b = v \cdot b_1$, где v --- обратим
    
    $a_1 = u^{-1} a = u^{-1} bc = u^{-1} vb_1c = (u^{-1}vc)b_1 \implies a_1 \divby b_1$
    
    
  \end{enumerate}
\end{proof}


\begin{definition}
  Пусть $A$ --- область целостности с единицей. Элемент $p \in A$ называется неразложимим или простым, если его нельзя представить в виде $p = ab$, где a, b --- необратимы.
\end{definition}

\begin{eg}
  \begin{enumerate}
    \item в $\Z:$ $\pm p$, где p --- простое
  \end{enumerate}
\end{eg}

\begin{definition}
  Пусть $K$ --- поле, $A = K[x]$. неразложимый в $K[x]$ многочлен называется неприводимым над $K$ (или в $K[x]$)
\end{definition}

\begin{eg}
  
  $\newline$

  \begin{enumerate}
    \item $x^2 + 1$ приводим над $\mathbb{C}$, неприводим над $\R$
    \item $x^2 - 2$ приводим над $\R$, неприводим над $\Q$
  \end{enumerate}
\end{eg}

\begin{definition}
  Пусть $A$ --- область целостности, $a, b \in A$. Элемент $d \in A$ называется НОД(a, b), если:

  \begin{enumerate}
    \item $a, b \divby d$
    \item выполнено условие: $a, b \divby x \implies d \divby x$
  \end{enumerate}
\end{definition}

\begin{properties}
  
  $\newline$

  \begin{enumerate}
    \item Если d является НОД(a, b) и $d_1$ ассоциировано с $d$, то $d_1$ является НОД(a, b)
    \item Если $d_1, d_2$ являются НОД(a, b), то $d_1, d_2$ ассоциированы
  \end{enumerate}
\end{properties}


\begin{proof}
  
  $\newline$
  \begin{enumerate}
    \item $a, b \divby d \implies a, b \divby d_1$
    
    $d \divby x \implies d_1 \divby x$

    \item $a, b \divby d$
    
    $\begin{cases}
      a, b \divby d_1 \\
      d \text{ - НОД(a, b)}
    \end{cases} \implies d \divby d_1$. Аналогично $d_1 \divby d$
  \end{enumerate}
\end{proof}

\begin{definition}
  Пусть $A$ --- область целостности с единицей. Элементы $a, b$ называются взаимно простыми, если 1 является НОД(a, b)
\end{definition}


\begin{definition}
  Кольцо $A$ называется факториальным, если $A$ --- область целостности с единицей и любой элемент $a \in A$ можно представить в виде произведения $a = u \cdot p_1 p_2 \ldots p_k$, где u обратим, $p_i$ неразложим

  Такое представление единственно с точностью до замены сомжножителей на ассоциированные, т.е. если $a = v q_1 \ldots q_m$ --- другое представление, то $k = m$ и можно так перенумеровать Элементы, что $\forall i: q_i$ ассоциирован с $p_i$
\end{definition}