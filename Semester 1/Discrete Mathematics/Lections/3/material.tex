
%\setcounter{chapter}{-1}



\lesson{3}{27.09.2023}{Продолжение}

\section{Перебор перестановок в лекцикографическом порядке (другой способ)}

% дискретка

\begin{algoritm}
    Будем брать элементы $(t_1, \ldots, t_k)$ из $T_k$ и сопоставлять им перестановки так, как делали ранее. Переход к следующей перестановке осуществляется путем прибавления единицы к $(t_1, \ldots, t_k)$. (причем последний элемент в $(t_1, \ldots, t_k)$ всегда ноль, т.к. ничего не значит)
\end{algoritm}

\begin{eg} (для $P_4$)
    \begin{tabular}{|c|c|c|}
        \hline
        num & t & p\\
        \hline
        0 & (0, 0, 0, 0) & (1, 2, 3, 4)\\
        1 & (0, 0, 1, 0) & (1, 2, 4, 3)\\
        2 & (0, 1, 0, 0) & (1, 3, 2, 4)\\
        3 & (0, 1, 1, 0) & (1, 3, 4, 2)\\
        4 & (0, 2, 0, 0) & (1, 4, 2, 3)\\
        5 & (0, 2, 1, 0) & (1, 4, 3, 2)\\
        6 & (0, 3, 0, 0) & (2, 1, 3, 4)\\
        $\vdots$ & $\vdots$ & $\vdots$\\
        23 & (3, 2, 1, 0) & (4, 3, 2, 1)\\
        \hline
    \end{tabular}
\end{eg}

\section{Перебор с минимальным изменением}
На каждой итерации будем менять только два соседних элемента. Для этого необходимо:
\begin{itemize}
    \item берем последний элемент в перестановке и меняем его с соседом до тех пор, пока элемент не дойдет до начала.
    \item когда этот элемент оказался в начале, мы меняем у него направление: теперь он будет менятся с соседом справ, а элемент, который оказался на последней позиции, делает 1 шаг (и так каждый раз, когда наш первый элемент меняет направление). Такие действия применяются ко всем элементам в перестановке.
\end{itemize}

\begin{algoritm}
    
    Кроме самой перестановки p и ее номера t (на этот раз младший разряд в номере --- последний), будем хранить массив d, в котором будем хранить направление движения элементов. Если элемент движется вправо, то d[i] = +, если влево, то d[i] = -. Начальное значение d[i] = - для всех i.

    Также храним $j$ где будем записывать индекс элемента в $t$, в котором значение увеличилось.

    \begin{enumerate}
        \item Прибавляем 1 к t
        \item Определяем номер разряда в котором значение увеличивается на 1, записываем в j
        \item $\forall i \in [1,n]: i > j$, меняем $d_i = -d_i$.
        \item j (не номер, именно такой элемент) меняем с соседом слева если $d_j=-$, и с соседом справа, если $d_j = +$.
    \end{enumerate}

\end{algoritm}

\begin{eg}
    \begin{tabular}{|c|c|c|c|c|}
        \hline
        num & t & d & p & j \\
        \hline
        0 & 0000 & $----$ & 1234 & - \\
        1 & 0001 & $----$ & 1243 & 4 \\
        2 & 0002 & $----$ & 1423 & 4 \\
        3 & 0003 & $----$ & 4123 & 4 \\
        4 & 0010 & $---+$ & 4132 & 3 \\
        5 & 0011 & $---+$ & 1432 & 4 \\
        6 & 0012 & $---+$ & 1342 & 4 \\
        $\vdots$ &$\vdots$ & $\vdots$ & $\vdots$ & $\vdots$ \\
        
        19 &0113& $--++$& 4231 & 4 \\
        20 &0120& $--++$& 4213 & 3 \\
        21 &0121& $--++$& 2413 & 4 \\
        22 &0122& $--++$& 2143 & 4 \\
        23 &0123& $--++$& 2134 & 4 \\
        24 &1000& $-+--$& --- & 1 \\
        \hline
        
        
        
    \end{tabular}
\end{eg}
\section{Сочетания и бином Ньютона}


$C_n^k = \frac{n!}{k!(n-k)!}$

\begin{properties} (Свойства сочетаний)
    \begin{enumerate}
        \item $C_n^k = C_n^{n-k}$
        \item $C_{n-1}^k + C_{n-1}^{k-1} = C_n^k$ 
    \end{enumerate}
\end{properties}

\begin{proof}
    Доказывается путем подстановки непосредственно в формулу, или можно рассматривать пути на целочисленной решетке. 
\end{proof}


\begin{theorem} (Бином Ньютона)
    
    $$\forall a, b \in \R, n \in \N_0: (a + b)^n = \sum_{k = 0}^{n}C_n^k a^k b^{n-k}$$
\end{theorem}

\begin{proof} (По индкуции)
    \begin{enumerate}
        \item База $n = 1$ очевидна.
        \item индкуционый переход $n - 1 \to n$:
        
        \[(a + b)^n = (a + b)(a + b)^{n-1} = a(a + b)^{n-1} + b(a + b)^{n-1} =\]
        \[= a \cdot \sum_{k=0}^{n-1}C_{n-1}^k a^k b^{n-1-k} + b \cdot \sum_{k=0}^{n-1}C_{n-1}^k a^k b^{n-1-k} = \]
        \[= \sum_{k=0}^{n-1}C_{n-1}^k a^{k+1} b^{n-1-k} + \sum_{k=0}^{n-1}C_{n-1}^k a^k b^{n-k} =\]
        \[= \sum_{k=1}^{n}C_{n-1}^{k-1} a^k b^{n-k} + \sum_{k=0}^{n-1}C_{n-1}^k a^k b^{n-k} =\]
        \[= a^n + \sum_{k=1}^{n-1}C_{n-1}^{k-1} a^k b^{n-k} + b^n + \sum_{k=1}^{n-1}C_{n-1}^k a^k b^{n-k} =\]
        \[= a^n + b^n + \sum_{k=1}^{n-1}(C_{n-1}^k + C_{n-1}^{k-1})a^k b^{n-k} = (a + b)^n \]
    \end{enumerate}
\end{proof}


\section{Перебор сочетаний с хорошей нумерацией}

Для того чтобы присваивать номер сочетанию, будем рассматривать сочетание как вектор из нулей и единиц: если элемент взяли --- единица, иначе --- ноль.

\begin{eg}
    Вектору $b = (1, 0, 1, 1, 1, 0, 0, 0)$ соответствует сочетание 1345.
\end{eg}
\begin{algoritm}
    определяем номер рекурсивно:

    $num(b[1:n-1], m) = \begin{cases}
        num(b[1:n-1], m), & \text{если} b[n] = 0,\\
        num(b[1:n-1], m-1), & \text{если} b[n] = 1,
    \end{cases}$

    Где $m$ --- кол-во единиц.
\end{algoritm}

\begin{eg}
    Рассмотрим $b = (1, 0, 1, 1, 1, 0, 0, 0), m = 4$:

    $\newline$

    $\begin{aligned}
        num(b, m) =& C_{6}^{4} + num(b[1:n-1], 3) = \\
        =& C_{6}^{4} + C_{4}^{3} + num(b[1:n-2], 3) = \\
        =& C_{6}^{4} + C_{4}^{3} + num(b[1:n-3], 2) = \\
        =& C_{6}^{4} + C_{4}^{3} + num(b[1:n-4], 2) = \\
        =& C_{6}^{4} + C_{4}^{3} + C_{2}^{2} + num(b[1:n-5], 1) = \\
        =& C_{6}^{4} + C_{4}^{3} + C_{2}^{2} + 0 = 15 + 4 + 1 = 20 \\
    \end{aligned}$
\end{eg}



