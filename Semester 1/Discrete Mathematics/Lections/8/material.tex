\lesson{8}{1.11.2023}{Энтропия. Сортировки.}

\begin{theorem}
    \label{theorem:entropy}
    Если функция $G(p_1, \ldots, p_m)$ обладает свойствами (1)-(6), то $G(p_1, \ldots, p_m) = H(P_m)$
\end{theorem}

\begin{lemma}
    Пусть $g(m) = G(\frac{1}{m}, \ldots, \frac{1}{m})$ и $g(m)$ обладает свойствами (1)-(6), тогда:
    \[g(m) = \log_2 m\]
\end{lemma}

\begin{proof}
    Возьмем 2 схемы: $Q_k$ и $Q_l$ равновероятных исходов:

    $$g(kl) = g(k) + g(l) \text{ -- из (6)} \implies g(m^k) = k \cdot g(m)$$

    $$g(2) = 1 \text{ -- из (3)} \implies g(2^s) = s$$

    Возьмем $s: 2^s \leq m^k \leq 2^{s+1}$, тогда, в силу монотонности $g$:

    \[g(2^s) \leq g(m^k) \leq g(2^{s+1}) \implies s \leq k g(m) \leq s + 1\]

    \[\frac{s}{k} \leq g(m) \leq \frac{s}{k} \frac{1}{k}\]

    \[\frac{\lfloor \log_ 2 m^k \rfloor}{k} \leq g(m) \leq \frac{\lfloor \log_ 2 m^k \rfloor + 1}{k}\]

    \[0 \leq g(m) - \log_2 m + \frac{\{k \log_2 m\}}{k} \leq \frac{1}{k}\]

    \[\lim_{k \to \infty} \frac{\{k \log_2 m\}}{k} = 0 \implies g(m) = \log_2 m\]
\end{proof}


\begin{lemma}
    Если $\forall p_i \in \Q: p_i \in P_m$, то:

    \[G(P_m) = H(P_m) \text{, где } G(P_m) \text{ удовлетворяет (1)-(6)}\]
\end{lemma}
\begin{proof}
    Пусть $p_i = \frac{r_1}{n}$. Пусть есть равновероятные схемы $Q_{r_1}, \ldots, Q_{r_n}$, тогда, 
    комбинируя их с $P_m$, получим равновероятную схему $Q_{n}$ С $n$ равновероятными исходами.
    
    \[G(Q_n) = G(P_m) + \sum_{i=1}^{n} p_i G(Q_{r_i})\]

    По лемме 1:
    \[\log_2 n = G(P_m) + \sum_{i=1}^{m} p_i \log_2 r_i\]

    \[G(P_m) = \log_2 n - \sum_{i=1}^{m} p_i \log_2 r_i = \log_2 n - \sum_{i=1}^{m} p_i( \log_2 p_i + \log_2 n) =\]
    
    \[= \log_2 n - \sum_{i=1}^{m} p_i \log_2 p_i - \sum_{i=1}^{m} p_i \log_2 n = \]
    
    \[= \sum_{i=1}^{m} p_i \log_2 \frac{1}{p_i}\]
\end{proof}

\begin{proof}(теоремы \ref{theorem:entropy})
    Для любого набора вероятностей $\{p_1,...,p_m\}$ рассмотрим сходящуюся к нему 
    последовательность рациональных наборов $\{p_1^{(k)},...,p_m^{(k)}\}$. 
    По лемме 2 для каждого из этих наборов $G=H$. Так как обе функции непрерывны, то это равенство выполняется и в предельной точке.
\end{proof}

\chapter{Сортировки}

\section{Сортировка слиянием (Фон Неймана)}
Первоначально сортируемый массив представляется в виде n “отсортированных массивов” длины 1. Далее массивы сливаются попарно и сортируются. Затем еще раз и т. д.

\begin{eg}
    для массива $a = [38, 27, 43, 3, 9, 82, 10]$:

    \begin{tabular}{ccccccc}
        |38| & |27| & |43|& |3|& |9|& |82|& |10|\\
        |27,& 38| & |3, & 43| & |9, & 82| & |10| \\
        |3,& 27,& 38,& 43|& |9,& 10,& 82| \\
        |3,& 9,& 10,& 27,& 38,& 43,& 82| \\
    
    \end{tabular}       
\end{eg}

\section{Сортировка вставками и сортировка Шелла}

Идея для сортировки вставками: наращивать отсортированную часть последовательности. Сначала берем уже “отсортированную” последовательность из одного элемента. Далее берем очередной элемент, сравниваем его с предыдущим и переставляем до тех пор, пока он не займет свое место.

\begin{eg}
    для массива $a = [38, 27, 43, 3, 9, 82, 10]$:

    \begin{tabular}{ccccccc}
        38 & 27 & 43 & 3 & 9 & 82 & 10 \\
        27 & 38 & 43 & 3 & 9 & 82 & 10 \\
        27 & 38 & 43 & 3 & 9 & 82 & 10 \\
        3 & 27 & 38 & 43 & 9 & 82 & 10 \\
        3 & 9 & 27 & 38 & 43 & 82 & 10 \\
        3 & 9 & 27 & 38 & 43 & 82 & 10 \\
        3 & 9 & 10 & 27 & 38 & 43 & 82 \\
        3 & 9 & 10 & 27 & 38 & 43 & 82 \\
    \end{tabular}
\end{eg}

Идея для сортировки Шелла: давайте попробуем сократить количество перемещений элементов за счет
того, что будем сдвигать их не на одну позицию, а сразу на несколько.

\begin{eg}
    Наш массив: 13  44  7  21  78  3  25  9  28  35  10  66  33  16
    Выберем шаг d = 6, отдельно сортируем (13  25  33), (44  9  16), (7  28), (21  35), (78 
    10) и (3  66).
    Затем уменьшим шаг d вдвое и повторим процедуру. На последнем шаге d = 1, что
    соответствует «обычной» сортировке вставками.
\end{eg}