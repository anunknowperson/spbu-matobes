

%\setcounter{chapter}{-1}


\chapter{Алгоритмы}

\lesson{1}{27.09.2023}{Продолжение}

\section{Продолжение}

% дискретка

\begin{enumerate}
    \item Прибавляем 1 к t
\end{enumerate}

% pic 1

$T_k = M_1 \times M_2 \times \dots \times M_k$

$|M_i| = j$

$(r_1, r_2, \dots, r_k)$

$T_k \leftrightarrow P_k$

% pic 2

\begin{enumerate}
    \item Прибавляем 1 к t
    \item Определяем номер разряда в котором значение увеличивается на 1, записываем в j
    \item Для любого i от 1 до N такого что i > j, меняем $d_i = -d_i$.
    \item j (не номер, именно такой элемент) меняем с соседом слева если $d_j=-$, и с соседом справа, если $d_j = +$.
\end{enumerate}

% pic 3

\section{Перебор и нумерации, сочетания}


$C_n^k = \frac{n!}{k!(n-1)!}$

\begin{enumerate}
    \item $C_n^k = C_n^{1-k}$
    \item $C_{n-1}^m + C_{n-1}^{m-1} = C_n^m$ 
\end{enumerate}


$(a + b)^n = \sum_{k = 0}^{n}C_n^k a^k b^{n-k}$


\begin{enumerate}
    \item $a = b = 1$

$2^n = \sum_{k = 0}^{n}C_n^k$

    \item $a = 1$, $b = -1$
    
\end{enumerate}


$(a + b)^n = \sum_{k = 0}^{n}C_n^k a^k b^{n-k}$

$(a + b)^n = (a+b)(a+b)^{n-1} = $
$a(a +b)^{n- 1} + (a + b)^{n - 1} = a \cdot \sum_{k = 0}^{n - 1}C_{n-1}^k a^k b^{n-1-k} +$

$+ b \cdot \sum_{k = 1}^{n-1}C_{n-1}^k a^k b^{n-1-k} = \sum_{k = n}^{n - 1} C_{n-1}^k a^{k + 1} b^{n-1-k} +$

$+ \sum_{k=n}^{n - 1} C_{n-1}^k a^k b^{n-k} = $

$\sum_{k = 1}^{n} C_{n-1}^{k-1} a^k b^{n-k} +$

$+ \sum_{k = 0}^{n - 1} C_{n-1}^k a^k b^{n-k} =$

$= a^n + \sum_{k = 1}^{n - 1} C_{n - 1}^{k-1} a^k b^{n-k} + \sum_{k = 1}^{n - 1} C_{n-1}^k a^k b^{n-k} + b^n$

$= a^n + \sum_{k = 1}^{n - 1}(C_{n-1}^{k-1} + C_{n-1}^k) a^k - b^{n-k} + b^n = $

% Не дописано? на доске конца не нашёл.

\begin{enumerate}
    \item Увеличиваем на 1 номер самого правого элемента который можно увеличить
    \item Справа выписываем натуральный ряд
\end{enumerate}

%pic 5

%pic 6

