
%\setcounter{chapter}{-1}


\chapter{Комбинаторика}

\lesson{1}{13.09.2023}{Введение}

\section{Основные определения}

\begin{definition}
    Перестановкой называется упорядоченный набор неповторяющихся элементов длины $n$, состоящий из элементов от $1$ до $n$.

    Число перестановок: $P_n = n!$
\end{definition}

\begin{definition}
    Размещением называется упорядоченный набор неповторяющихся элементов длины $k$, состоящий из элементов от $1$ до $n$.

    Число размещений: $A_n^k = n(n - 1)(n - 2) \cdot \ldots \cdot (n - k + 1) = $
    
    $= \frac{n(n - 1)(n - 2) \cdot \ldots \cdot (n - k + 1)(n - k)!}{(n - k)!}= \frac{n!}{(n-k)!}$
\end{definition}


\begin{definition}
    Сочетанием называется набор неповторяющихся элементов длины $k$, состоящий из элементов от $1$ до $n$.

    Число сочетаний: $C_n^k = \frac{A_{n}^{k}}{k!}=\frac{n!}{k!(n-k)!}$
\end{definition}


\begin{definition}
    Перестановки с повторениями: $\overline{P_n} = \frac{n!}{n_1! \cdot n_2! \cdot \ldots \cdot n_k!}$
\end{definition}

\begin{definition}
    Размещения с повторениями: $\overline{A_n^k} = n^k$
\end{definition}

\begin{definition}
    Сочетания с повторениями: $\overline{C_n^k} = C_{n + k - 1}^k$
\end{definition}

\begin{eg} (Толкование к сочетаниям с повторениями)
    Сколькими способами можно разложить пять одинаковых шаров по трём различным ящикам? На число шаров в ящике ограничений нет.

    Решение:

    Представим себе, что ящики стоят вплотную друг к другу. Три таких ящика --- это
    фактически две перегородки между ними. Обозначим шар нулём, а перегородку --- единицей.
    Тогда любому способу раскладывания пяти шаров по трём ящикам однозначно соответствует
    последовательность из пяти нулей и двух единиц; и наоборот, каждая такая последовательность
    однозначно определяет некоторый способ раскладывания. Например, 0010010 означает, что в
    первом ящике лежат два шара, во втором --- два шара, в третьем --- один шар; последовательность 0000011 соответствует случаю, когда все пять шаров лежат в первом ящике.

    Теперь ясно, что способов разложить пять шаров по трём ящикам существует ровно столько
же, сколько имеется последовательностей из пяти нулей и двух единиц. А число таких последовательностей равно $C_{7}^{2}$
\end{eg}

\section{Множества}

\begin{theorem} (Формула включений-исключений)
    
    $|A_1 \cup A_2 \cup \ldots \cup A_n| = \left|\bigcup_{i=1}^{n} A_i\right| =$

    $$  = \sum_{i=1}^{n} |A_i| -\sum_{1 \leq i < j \leq n} |A_i \cap A_j| + \sum_{1 \leq i < j < k \leq n} |A_i \cap A_j \cap A_k| - \ldots + (-1)^{n-1} |A_1 \cap A_2 \cap \ldots \cap A_n| $$
\end{theorem}

\begin{proof} (докажем по индукции)
    \begin{enumerate}
        \item База индукции: $n = 2: |A_1 \cup A_2| = |A_1| + |A_2| - |A_1 \cap A_2|$
        \item Переход индукции: $n \to n + 1:$
        
        $|A_1 \cup A_2 \cup \ldots \cup A_{n+1}| = |A_1 \cup A_2 \cup \ldots \cup A_n| + |A_{n+1}| - |(A_1 \cup A_2 \cup \ldots \cup A_n) \cap A_{n+1}| = $

        $$= \sum_{i=1}^{n} |A_i| -\sum_{1 \leq i < j \leq n} |A_i \cap A_j| + \sum_{1 \leq i < j < k \leq n} |A_i \cap A_j \cap A_k| -$$ 
        $-\ldots + (-1)^{n-1} |A_1 \cap A_2 \cap \ldots \cap A_n|$ + $|A_{n+1}| - |(A_1 \cap A_{n+1}) \cup (A_2 \cap A_{n+1}) \ldots \cup (A_n \cap A_{n+1})| = $

        $$= \sum_{i=1}^{n} |A_i| -\sum_{1 \leq i < j \leq n} |A_i \cap A_j| + \sum_{1 \leq i < j < k \leq n} |A_i \cap A_j \cap A_k| -$$
        
        $$- \ldots + (-1)^{n-1} |A_1 \cap A_2 \cap \ldots \cap A_n| - (\sum_{i=1}^{n}|A_i \cap A_{n+1}| -$$
        
        $$- \sum_{1 \leq i < j \leq n}|A_i \cap A_j \cap A_{n+1}| + \ldots (-1)^{n-1}|A_1 \cap A_2 \cap \ldots \cap A_{n+1}|) = $$

        $$= \sum_{i=1}^{n+1} |A_i| -\sum_{1 \leq i < j \leq n+1} |A_i \cap A_j| + \sum_{1 \leq i < j < k \leq n+1} |A_i \cap A_j \cap A_k| - $$
        $\ldots + (-1)^{n} |A_1 \cap A_2 \cap \ldots \cap A_{n+1}|$
    \end{enumerate}
\end{proof}

\section{Разбиения}

\begin{definition}
    Пусть $A$ --- множество. Имеется $A_1, A_2, \ldots, A_n$. 
    
    Совокупность этих множеств --- разбиение, если: $A = \bigcup_{i=1}^{n} A_i; A_i \cap A_j = \varnothing$
\end{definition}

\begin{definition}
    Пусть у $A$ есть разбиения $\mathcal{A}$ и $\mathcal{B}$
    
    Тогда $\mathcal{B}$ --- измельчение $\mathcal{A}$, если $\forall B_i \in \mathcal{B} \; \exists! A_j \in \mathcal{A}: B_i \subset A_j$
\end{definition}

\begin{definition}
    Произведение разбиений --- разбиение, которое является измельчением $\mathcal{A}$ и $\mathcal{B}$ и является самым крупным измельчением.
\end{definition}