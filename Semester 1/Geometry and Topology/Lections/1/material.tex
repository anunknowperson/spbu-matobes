\lesson{1}{09.09.2023}{Векторное пространство}

\chapter{Векторное пространство}

\section{Определение векторного пространства}

\begin{definition}
    Пусть $V$ - множество; \newline
    $+: V \times V \longrightarrow V
    \newline
    \cdot : \mathbb{R} \times V \longrightarrow V
    \newline
    \forall u, w, v \in V: \forall \alpha , \beta$
    \begin{enumerate}
        \item (u+v)+w=(u+v)+w (ассоциативность сложения)
        \item u+v=v+u (коммутативность сложения)
        \item $\exists ! 0 \in V: u+0=0+u=u$ (нейтральный элемент по сложению)
        \item $\exists u; -u: u+(-u)=0$ (обратный элемент по сложению)
        \item $\alpha (u+v)=\alpha u+\alpha v$ (дистрибутивность)
        \item $(\alpha \cdot \beta)u=\alpha(\beta \cdot u)$ (ассоциативность умножения)
        \item $1 \cdot u=u$ (нейтральный элемент по умножению)
    \end{enumerate}
    Если 1-8 выполняются, то V - (вещественное) векторное пространство.
\end{definition}

\begin{eg}
    \begin{enumerate}
        \item $\mathbb{R}^n=\mathbb{R}\times \mathbb{R}\times ... \times \mathbb{R}$ - n-мерное пространство ($a_{1}...a_{n}$)+($b_{1}...b_{n})=(a_{1}+b_{1}...a_{n}+b_{n})$
        \item Множество многочленов V \newline
         Множество многочленов n степени --- не веркторное пространство, т. к. ($x^n+1)+(-x^n+x)=x+1$ --- сложение не определено
        \newline Множество многочленов степени $n\leqslant n$ --- векторное пространство.
        \item Множество определенных на $[a .. b]$, непрерывных и имеющих непрерывную производную функций --- векторное пространство.
        \item Матрицы $n\times m$ --- векторное пространство.
        \item Множество вращений шара (сложение --- композиция, умножение --- умножение угла на число на число) --- не векторное пространство. (Упражнение: докажите почему)
    \end{enumerate}
\end{eg}

\begin{property} (Доказуемые свойства)
    \begin{enumerate}
        \item $\overline{1}$ --- единственный.
        \item $\begin{cases}
            u + v = 0 \\
            u + w = 0
        \end{cases} \implies v = w$
        \item  $-\overline{1} \cdot u = -u$
        \item $u \cdot 0 = 0$
    \end{enumerate}
\end{property}

\section{Линейная комбинация, линейная зависимость и  линейная независимость}
\begin{definition}
    $V$ - векторное пространство и векторы \\ $v_1,v_2,v_3,..., v_n \in V$.
    Система $v_1,...,v_n$ называется линейно независимой (ЛНЗ), если из
    $\alpha_1 v_1 + \alpha_2 v_2 + ... + \alpha_n v_n = 0 \implies \alpha_1=\alpha_2=...=\alpha_n =0$.
\end{definition}

\begin{definition}
    Если $\alpha_1,..., \alpha_n \in \R$, $v_1,...,v_n \in V$.
    То $\alpha_1 v_1 + \alpha_2 v_2 + ... + \alpha_n v_n$ -- линейная комбинация (ЛК)
    векторов $v_1,...,v_n$.
\end{definition}

\begin{definition}
    Если $\exists \alpha_1,..., \alpha_n$, не все $=0$, но $\alpha_1 v_1 + \alpha_2 v_2 + ... + \alpha_n v_n = 0$,
    то система $v_1,...,v_n$ называется линейно зависимой (ЛЗ).
\end{definition}

\begin{theorem}
    $v_1,...,v_n$ -- ЛЗ $\Leftrightarrow$ один из этих векторов можно представить как ЛК остальных.
    $\exists i: v_i= \alpha_1 v_1 +\alpha_2 v_2 + ... + \alpha_{i-1} v_{i-1} + \alpha_{i+1} v_{i+1} + ... +\alpha_n v_n$
\end{theorem}
\begin{proof}
    $\implies : \exists \alpha_1,...,\alpha_n (\exists i: \alpha_i \neq 0)$
    \begin{gather*}
        \alpha_1 v_1 + \alpha_2 v_2 + ... + \alpha_n v_n = 0\\
        \alpha_i v_i = - \alpha_1 v_1 - \alpha_2 v_2 - ... - \alpha_{i-1} v_{i-1} - \alpha_{i+1} v_{i+1} - ... -\alpha_n v_n\\
        \alpha_i \neq 0 \quad v_i = -\frac{\alpha_1}{\alpha_i}v_1 -... - \frac{\alpha_n}{\alpha_i}v_n\\
        \Leftarrow: v_i = \alpha_1 v_1 + ... + \alpha_n v_n \text{ без } i\text{-ого слагаемого}\\
        \alpha_1 v_1 + \alpha_2 v_2 + ... + (-1)v_i + ... + \alpha_n v_n =0\\
        \text{ЛК } = 0 \text{ не все коэффициенты} = 0
    \end{gather*}
\end{proof}

\begin{prop}
    $v_1,...,v_n $ -- ЛНЗ, то любой его поднабор тоже ЛНЗ.\\
    $v_1,...,v_n $ -- ЛЗ, то при добавлении векторов, набор останется ЛЗ.
\end{prop}


\begin{theorem}
    $v_1,..., v_n$ -- ЛНЗ $\Leftrightarrow$ если
    \begin{gather*}
        \alpha_1 v_1 + ... + \alpha_n v_n = \beta_1 v_1 + ... + \beta_n v_n\\
        \implies \alpha_1 = \beta_1; \alpha_2 = \beta_2; ... ; \alpha_n = \beta_n
    \end{gather*}
\end{theorem}
\begin{proof}
    \begin{gather*}
        (\alpha_1 - \beta_1)v_1 + (\alpha_2 - \beta_2)v_2 + ...
        + (\alpha_n - \beta_n)v_n = 0\\
        \alpha_i - \beta_i = 0 \Leftrightarrow v_1, ..., v_n\text{-- ЛНЗ}
    \end{gather*}
\end{proof}